\section{Problema} 

\subsection{Critérios}
	O problema que se quer resolver é o seguinte: \\
	Implementar um orquestrador que para um número $N$ de elevadores e $M$ andares, tem que cumplir com os seguintes critérios:
	\begin{itemize}
		\item Redução de consumo de energia: medido pelo percurso total de cada elevador
		\item Aumento do conforto e satisfação dos usuários: medido pelo tempo de espera e pelo tempo dentro do elevador
		\item Combinação dos dois critérios anteriores
	\end{itemize}
	
\subsection{Definições previas}
	Esta implementação deve usar heurísticas para cumplir com os critérios anteriores. Então para entender 		
	\begin{description}
		\item [Heurística] Método com o objetivo de encontrar soluções para um problema, ainda suas respostas não sempre são ótimas.
		\item[Função objetivo] Função utilizada por uma heurística como critério de comparação.
		\item[Espaço de búsqueda] Conjunto de estados possíveis desde onde a heurística está em certo momento.
	\end{description}