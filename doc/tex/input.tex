\section{Dados de entrada}

O orquestrador vai receber os dados de entrada dum archivo de dados gerado aleatoriamente que conterá as chamadas dos elevadores e os tempos de estas. A continuação se explicará os aspectos relevantes dos dados.

\subsection{Restrições}
\label{subsec:restricoes}
	O archivo de entrada vai ter a seguinte forma: \\
		\begin{lstlisting}
			|${T}$|
			|${NC}_1$|
			|${{in1}_1} {{out1}_1}$|
			|${{in1}_2} {{out1}_2}$|
			...
			|${{in1}_{{NC}_1}} {{out1}_{{NC}_1}}$|
			.
			.
			.
			|${NC}_T$|
			|${{inT}_1} {{outT}_1}$|
			|${{inT}_2} {{outT}_2}$|
			...
			|${{inT}_{{NC}_T}} {{outT}_{{NC}_T}}$|
		\end{lstlisting}
	A primeira linha tem $T$ que será o tempo para o orquestrador, ou seja que vai funcionar no intervalo $[ 0 , T ]$. Logo se tem $T$ grupos, onde para o grupo $t$, ${NC}_t$ é o número de chamadas dos elevadores no tempo $t$, seguido de ${NC}_t$ linhas descrevendo cada uma das chamadas como ${in} {out}$ onde ${in}$ é o andar onde a pessoa está e ${out}$ o andar onde a pessoa quer ir.
	Em quanto as restrições numéricas temos:
	\begin{itemize}
		\item $N$ será o número de elevadores e $M$ o número de andares
		\item ${in}$ e ${out}$ sempre estão no intervalo $[1,M]$ e não podem ser iguaís
		\item O número máximo de chamadas para um determinado $t \in [0,T]$ é ${max_{NC}}$
		\item Todos os elevadores tem a mesma capacidade $C$
		\item Todos os elevadores servem todos os andares
	\end{itemize}
	
\subsection{Gerador aleatório de dados}
	O gerador de dados está escrito na linguagem de programação Java e recebe 4 parâmetros: $T$ , ${max}_{NC}$ , $M$ e o nome para o archivo gerado.
	A função principal que gera uma soa chamada é a seguinte:
	\begin{lstlisting}
		public ElevatorCall generateSingleElevatorCall( Integer currentTime ){
			Integer in = Utils.randomBetween( 1 , numFloors ) ;
			Integer out = Utils.randomBetween( 1 ,  numFloors ) ;
			while( out == in ) out = Utils.randomBetween( 1 ,  numFloors ) ;
			return new ElevatorCall( in , out , currentTime ) ;
		}
	\end{lstlisting}
	Esta é usada por a função:
	\begin{lstlisting}
		public List<ElevatorCall> generateElevatorCalls( Integer currentTime ){
			Integer numCalls = Utils.randomBetween( 1 ,  maxNumCalls ) ;
			List<ElevatorCall> lstCalls = new ArrayList<ElevatorCall>() ;
			for( Integer i = 0 ; i < numCalls ; i++)
				lstCalls.add( generateSingleElevatorCall( currentTime ) ) ;
			return lstCalls ;
		}
	\end{lstlisting}
	Esta última função é usada $T$ vezes para gerar todas as chamadas e guardá-las no archivo especificado como parámetro. \\
	O gerador de dados está no paquete Utils, no archivo CallGenerator.java do projeto.